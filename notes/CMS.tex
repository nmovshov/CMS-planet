\documentclass[amsmath,amsfonts,rmp,letterpaper]{revtex4}
\usepackage{graphicx}
\usepackage{amsthm}
\usepackage{silence}
% ----------------------------------------------------------------
\vfuzz2pt % Don't report over-full v-boxes if over-edge is small
\hfuzz2pt % Don't report over-full h-boxes if over-edge is small
% NUMBERING ------------------------------------------------------
%\numberwithin{equation}{section}
% MATH MODE COMMANDS ---------------------------------------------
\newcommand{\V}[1]{\mathbf{#1}} % bold vector or matrix
\newcommand{\sub}[1]{_{\text{#1}}} % text subscript
\newcommand{\di}{\partial} % partial derivative
\newcommand{\unit}[1]{\;\mathrm{#1}} % attach units
\newcommand{\kros}{\times} % vector product
\newcommand{\grad}{\boldsymbol{\nabla}} % gradient operator
\newcommand{\divergenz}{\boldsymbol{\nabla}\cdot} % divergence operator
\newcommand{\curl}{\boldsymbol{\nabla}\kros} % curl (rot) operator
\newcommand{\abs}[1]{\left\vert#1\right\vert} % generic absolute value
\newcommand{\set}[1]{\left\{#1\right\}} % put elements between { }
\newcommand{\Real}{\mathbb R} % Real numbers field
\newcommand{\Cmplx}{\mathbb C} % Complex numbers field
\newcommand{\eps}{\varepsilon} % variirtes epsilon
\newcommand{\mean}[1]{\langle #1 \rangle} % physics mean
\newcommand{\about}{\sim\!}
\newcommand{\arr}{\V{r}}
\newcommand{\arp}{\V{r'}}
\newcommand{\om}{\omega}
\renewcommand{\inf}{\infty}
\newcommand{\const}{\mathrm{const.}}
\newcommand{\red}[1]{\textbf{\textcolor{red}{#1}}}
%\DeclareMathOperator{\XXX}{xxx} % \int like operators
%\DeclareMathOperator*{\XXX}{xxx} % \lim like operators
% HYPER REF OPTIONS ----------------------------------------------
\usepackage[dvipsnames,usenames]{color}
\usepackage[pdftex,bookmarks=false,pdfstartview=FitH,colorlinks]{hyperref}
\hypersetup{linkcolor=Sepia,citecolor=Sepia}% additional options
% LENGTH PARAMETERS ----------------------------------------------
\addtolength{\textwidth}{-0.8in}% the default in one column is too wide
\addtolength{\oddsidemargin}{0.4in}
%\setlength{\parskip}{0.5ex plus 0.2ex minus 0.2ex}
% ----------------------------------------------------------------
\begin{document}

\title{Concentric Maclaurin Spheroids: theory and practice}%
\author{}
\begin{abstract}
Summary of the theory and practice of modeling rotating fluid planets by Hubbard's
Concentric Maclaurin Spheroids technique. These notes provide the mathematical
basis for using gravity measurements to learn about planetary interiors.
Mathematical statements are checked to my satisfaction unless otherwise noted
(important exception is the addition theorem). Primary references are
\citep{Zharkov1978} for the mathematical foundation, and \citep{Hubbard2013} and
Bill's personal notes for the CMS theory.
\end{abstract}
\maketitle
\tableofcontents

\section{Definitions and notation}\label{sec:definitions}
Given a density distribution $\rho(\arp)$ inside the planet, the \emph{total
potential} is defined as
\begin{equation}
U(\arr) = V(\arr) + Q(\arr)
\end{equation}
where
\begin{equation}\label{eq:gravpot}
V(\arr) = G\int{\rho(\arp)/\abs{\arr - \arp}\,d\tau'}
\end{equation}
is the gravitational potential and
\begin{equation}\label{eq:cetrifugal_potential}
Q(\arr) = \frac{1}{2}\omega^2r^2\sin^2\theta
\end{equation}
is the centrifugal potential; we will use $d\tau$ for a volume element, $\omega$
is the angular rotation velocity (assumed constant and usually about a principal
axis), $r=\abs{\arr}$, and $\theta$ is the angle from the rotation axis.
\emph{Note that Hubbard and Zharkov and Trubitsyn use an unusual positive
potential presumably for algebraic convenience. This also means accelerations are
given by the positive gradient of potential.}

When the planet is in hydrostatic equilibrium the level surfaces of potential are
also level surfaces of pressure and of density. Including the free surface where
the pressure $p=0$. This can be shown rigorously \citep[e.g.][]{Batchelor1967} and
is also fairly intuitive. Finding the shape of the planet is therefore reduced to
the problem of finding the level surfaces $U(\arr)=\text{constant}$. This problem
is easy to state but hard to solve, essentially because the volume of space over
which the integral above is taken is unknown and must be found as part of the
self-consistent solution.

Equilibrium figures which differ only slightly from spheres are called \emph
{spheroids}. A dimensionless parameter describing the importance of rotation is
\begin{equation}\label{eq:smallq}
q = \frac{\omega^2a^3}{GM}
\end{equation}
where $M$ is the planet's mass and $a$ is the \emph{equatorial radius}.
% section definitions (end)

\section{Decomposition of potential into spherical harmonics}
\label{sec:spherical_harmonics}
The expression for the gravitational potential, eq.~(\ref{eq:gravpot}) can be
written as a sum of powers of $r$ using the decomposition of $1/\abs{\arr-\arp}$
in Legendre polynomials (see \textsc{legendre.pdf}):
\begin{equation}\label{eq:lege_exp}
\begin{split}
\frac{1}{\abs{\arr-\arp}} &= \frac{1}{r\sqrt{1 - 2t(r'/r) + (r'/r)^2}} = \\
&=
\begin{cases}
\frac{1}{r}\sum_{n=0}^{\infty}(\frac{r'}{r})^nP_n(t),&\qquad r>r',\\
\frac{1}{r}\sum_{n=0}^{\infty}(\frac{r'}{r})^{-n-1}P_n(t),&\qquad r<r',
\end{cases}
\end{split}
\end{equation}
where $\gamma$ is the angle between the radius vectors $\arr$ and $\arp$ and
$t=\cos\gamma$. If $r>r'$ for all points where $\rho(\arp)>0$ (i.e.~inside the
planet) then the potential is called \emph{external}. The Legendre polynomials are
given by Rodrigues's formula:
\begin{equation}\label{eq:rodrigues}
P_n(x) = \frac{1}{2^nn!}\frac{d^n}{dx^n}(x^2 - 1)^n
\end{equation}
(although this is not the easiest way to obtain them explicitly) for
$x\in{[-1,1]}$ and luckily there is no confusion about normalization. It will be
useful to remember that
\begin{equation*}
P_0 = 1, \qquad P_2(t) = \frac{3}{2}t^2 - \frac{1}{2}
\end{equation*}
and that $P_n(1)=1$.
The gravitational potential in terms of Legendre polynomials is
\begin{equation}\label{eq:lege_gravity}
V(\arr) = \frac{G}{r}\sum_{n=0}^\infty\int\rho(\arp)P_n(t)(r'/r)^k\,d\tau'
\end{equation}
where
\begin{equation*}
k=
\begin{cases}
n,&\qquad r>r',\\
-(n+1),&\qquad r<r',
\end{cases}
\end{equation*}
and the integration is over the (as yet unknown) volume of the planet.

The expansion in Legendre polynomials is compact and neat but it is of little
utility because it does not separate terms arising from the mass distribution from
those arising from the location where the potential is to be evaluated. In
spherical polar coordinates the variable
\begin{equation*}
t = \cos\gamma = \cos\theta\cos\theta' + \sin\theta\sin\theta'\cos(\varphi -
\varphi')
\end{equation*}
mixes the colatitude $\theta$ and longitude $\varphi$ of the integration
variable and the point of measurement. The salvation comes from the \emph{addition
theorem for spherical harmonics} (\red{which I can't derive}):
\begin{equation}\label{eq:addition}
P_n(\cos\psi) = P_n(\cos\theta)P_n(\cos\theta') +
2\sum_{m=1}^{n}\frac{(n - m)!}{(n + m)!}P_n^m(\cos\theta)P_n^m(\cos\theta')
\cos\bigl[m(\varphi - \varphi')\bigr],
\end{equation}
with the associated Legendre functions
\begin{equation}\label{eq:assoc_lege}
P_n^m(x) = (-1)^m(1 - x^2)^{m/2}\frac{d^m}{dx^m}P_n(x).
\end{equation}
Unfortunately there is not a universal consensus on the exact form of the $P_n^m$
functions. If a different normalization is used then the following expressions
will all be somewhat different. With the aid of the addition theorem we can
decompose the potential as\WarningsOff\footnote{Note typo in Z\&T.}\WarningsOn
\begin{equation}\label{eq:spherical_gravity}
\begin{split}
V(\arr) = \frac{G}{r}&\Biggl(
\sum_{n=0}^\infty{}P_n(\cos\theta)\int_\tau{}\rho(\arp)P_n(\cos\theta')
(r'/r)^k\,d\tau' \\
&+ \sum_{n=1}^{\infty}\sum_{m=1}^{n}P_n^m(\cos\theta)\cos(m\varphi)\int_\tau\frac
{2(n - m)!}{(n + m)!}\rho(\arp)P_n^m(\cos\theta')\cos(m\varphi')\Bigl(\frac
{r'}{r}\Bigr)^k\,d\tau' \\
&+ \sum_{n=1}^{\infty}\sum_{m=1}^{n}P_n^m(\cos\theta)\sin(m\varphi)\int_\tau\frac
{2(n - m)!}{(n + m)!}\rho(\arp)P_n^m(\cos\theta')\sin(m\varphi')\Bigl(\frac
{r'}{r}\Bigr)^k\,d\tau'
\Biggr)
\end{split}
\end{equation}
with $k$ as before.

The above expansion is general, not requiring hydrostatic equilibrium or principal
axis rotation. If the planet is fluid then at equilibrium the rotation will
always be about a principal axis. If we take the polar axis (call it the $z$ axis)
to coincide with the rotation axis then symmetry requires that $\rho(\arr)$ and
therefor $V(\arr)$ cannot
depend on longitude $\varphi$ and must include only even
powers of $\cos\theta$ (for symmetry about the equator). In this case we write
a simpler expansion involving only ordinary Legendre polynomials of only even
degree:
\begin{subequations}\label{eq:simp_sphe_gravity}
\begin{equation}
V(r,\theta)=\frac{G}{r}\sum_{n=0}^{\infty}\left(r^{-2n}D_{2n} + r^{2n +
1}D'_{2n}\right)P_{2n}(\cos\theta)
\end{equation}
with
\begin{align}
D_n &= \int_{r'<r}\rho(\arp)(r')^nP_n(\cos\theta')\,d\tau',\\
D'_n &= \int_{r'>r}\rho(\arp)(r')^{-n-1}P_n(\cos\theta')\,d\tau'.
\end{align}
\end{subequations}
The coefficients $D_n$ are usually replaced with the non-dimensional coefficients
$J_n=D_n/(Ma^n)$.

% section spherical_harmonics (end)

\section{The external potential}\label{sec:external_potential}

If the potential is to be evaluated at a point exterior to the surface of the
planet then $r>r'$ for all differential volume elements in the integral
expressions above. The general form eq.~(\ref{eq:spherical_gravity}) can be
rearranged slightly and rewritten in a form more convenient for comparison with
measured quantities:
\begin{subequations}\label{eq:external_potential}
\begin{equation}
\begin{split}
V_e = \frac{GM}{r}\Bigg(1 &- \sum_{n=1}^{\infty}(a/r)^nJ_nP_n(\cos\theta)\;+ \\
&+\sum_{n=1}^{\infty}\sum_{m=1}^{n}(a/r)^nP_n(\cos\theta)
\Bigl[C_{nm}\cos(m\varphi) + S_{nm}\sin(m\varphi)\Bigr]\Biggr),
\end{split}
\end{equation}
with the coefficients
\begin{align}
Ma^nJ_n &= -\int\rho(\arp)(r')^nP_n(\cos\theta')\,d\tau' = -D_n,\\
Ma^nC_{nm} &= \frac{2(n - m)!}{(n + m)!}\int\rho(\arp)(r')^nP_n^m(\cos\theta')
\cos(m\varphi')\,d\tau',\\
Ma^nS_{nm} &= \frac{2(n - m)!}{(n + m)!}\int\rho(\arp)(r')^nP_n^m(\cos\theta')
\sin(m\varphi')\,d\tau'.
\end{align}
\end{subequations}
(Remember $M=\int\rho\,d\tau'$ is the planet's mass and $a$ is the equatorial
radius.) Different normalizations of the associated Legendre functions are
sometimes used leading to expansions with the same form but different meaning of
the $J_n$, $C_{nm}$, and $S_{nm}$ coefficients. There is no easy way to guard
against errors or guess which normalization was used if it is not explicitly
given. The decomposition can also be carried out with the complex form of the
Legendre functions leading to even more confusion.

A natural choice of reference frame can eliminate many of the coefficients in the
expansion~(\ref{eq:external_potential}). First, if the origin of coordinates is
chosen at the center of mass of the planet, $\V{R}=(X,Y,Z)$ then
\begin{align*}
-MaJ_1 &= \int{z'\,dm'} = MZ = 0,\\
MaC_{11} &= \int{x'\,dm'} = MX = 0,\text{ and}\\
MaS_{11} &= \int{y'\,dm'} = MY = 0
\end{align*}
so that $J_1$, $C_{11}$, and $S_{11}$ vanish. Next, we can relate the expansion
coefficients to the moments of inertia. We designate those:
\begin{subequations}
\begin{align}
B &= \int_\tau\rho(\arp)(x'^2 + z'^2)\,d\tau',\\
A &= \int_\tau\rho(\arp)(y'^2 + z'^2)\,d\tau',\\
C &= \int_\tau\rho(\arp)(x'^2 + y'^2)\,d\tau',
\end{align}
for the principal moments and
\begin{align}
D &= \int_\tau\rho(\arp)\,y'z'\,d\tau',\\
E &= \int_\tau\rho(\arp)\,x'z'\,d\tau',\\
F &= \int_\tau\rho(\arp)\,x'y'\,d\tau',
\end{align}
for the diagonal, so-called \emph{products of inertia}, also called
\emph{centrifugal moments}.
\end{subequations}
The relations with the expansion coefficients in eq.~(\ref{eq:external_potential})
become clear by writing out the degree 2 Legendre functions in Cartesian
coordinates. It is easy to show by direct comparison that
\begin{subequations}
\begin{equation}
-a^2MJ_2 = \frac{A + B}{2} - C,\quad\text{and}\quad a^2MC_{22} = \frac{B - A}{4},
\end{equation}
and also that
\begin{equation}
D = a^2MS_{21},\quad{}E = a^2MC_{21},\quad\text{and}\quad{}F = 2a^2MS_{22}.
\end{equation}
\end{subequations}
Usually $A=B<C$ so that $a^2MJ_2=(C-A)$. Also, if we align the coordinate axes
with the planet's principal axes of inertia then the centrifugal moments vanish
and
\begin{equation}
S_{21} = C_{21} = S_{22} = 0.
\end{equation}
And of course it is still true that for a fluid planet at equilibrium the density
must be independent of longitude and symmetrical about the equator and therefore
\begin{equation}
D_{2n+1} = J_{2n+1} = 0,\quad\forall{n}.
\end{equation}

Finally, expressing the centrifugal potential~(\ref{eq:cetrifugal_potential}) in
terms of $P_2$:
\begin{equation}
Q(\arr) = \frac{1}{3}\omega^2r^2\bigl[1 - P_2(\cos\theta)\bigr]
\end{equation}
and using the small parameter $q$ (\ref{eq:smallq}) the total external potential
is
\begin{equation}\label{eq:total_external_potential}
V_e(\arr) = \frac{GM}{r}\left[1 - (a/r)^2J_2P_2 - (a/r)^4J_4P_4 -
(a/r)^6J_6P_6 - \cdots + \frac{1}{3}(r/a)^3\bigl(1 - 
P_2(\cos\theta)\bigr)\,q\right].
\end{equation}

% section external_potential (end)

\section{A constant density spheroid} % (fold)
\label{sec:a_constant_density_spheroid}
In the special case where $\rho=\const$ the integrals in the definitions of the
gravity coefficients \eqref{eq:simp_sphe_gravity} are greatly simplified. In fact
this case there is a closed analytic solution showing that the equilibrium surface
is an ellipsoid with ellipticity related to the dimensionless rotation parameter
\begin{equation}
m = \frac{3\om^2}{4\pi{G}\rho}.
\end{equation}
We are not interested so much in the closed form solution itself but rather in
the simplified form of the gravity coefficients.

For the external potential of a rotating fluid in equilibrium we can write
eqs.~\eqref{eq:external_potential} in polar coordinates:
\begin{equation}
D_n = \rho\,\int_\tau (r')^nP_n(\cos\theta')\,d\tau' =
2\pi\rho\int_{0}^{\pi}d\theta'\int_{0}^{r(\theta')}(r')^nP_n(\cos\theta')
(r')^2\sin\theta'\,dr'
\end{equation}
and moving to the variable $\mu=\cos\theta$ we have
\begin{equation}
D_n = \frac{2\pi\rho}{n + 3}\int_{-1}^{1}d\mu'\,P_n(\mu')r(\mu')^{n + 3}.
\end{equation}

Moving now to the non-dimensional radius $\xi(\mu) = r(\mu)/a$ and remembering
that only even degree coefficients contribute we can write
\begin{equation}
Ma^nJ_n = -D_n =
-\frac{4\pi\rho{}a^{n + 3}}{n + 3}\int_{0}^{1}d\mu'\,P_n(\mu')\xi(\mu')^{n + 3}.
\end{equation}
And finally, remembering that
\begin{equation}
M = \frac{4\pi\rho{}a^3}{3}\int_{0}^{1}d\mu'\,\xi(\mu')^3,
\end{equation}
we have the general expression for $J_n$:
\begin{equation}\label{eq:Jn_maclaurin_ext}
J_n = -\frac{3}{n + 3}\frac{\int_{0}^{1}d\mu'\,P_n(\mu')\xi(\mu')^{n+3}}{\int_{0}^
{1}d\mu'\,\xi(\mu')^3}.
\end{equation}

We can now write an implicit equation for $r(\mu)$ by requiring it to be a level
surface, i.e.~to be an curve of constant potential. That is, constant \emph{total}
potential, including the centrifugal term. The total potential at a point on the
surface is (eq.~(\ref{eq:total_external_potential}))
\begin{equation}
U(r,\mu) = \frac{GM}{r} \Bigl[ 1 -
\sum_{k=1}^{\infty}\left(\frac{a}{r}\right)^{2k} J_{2k}P_{2k}(\mu) \Bigr] + 
\frac{1}{3}r^2\om^2[1 - P_2(\mu)].
\end{equation}
We require that the potential at any point on the surface equal the potential on
the equator:
\begin{equation}
U(a,0) = \frac{GM}{a} \Bigl[1 - \sum_{k=1}^{\inf} J_{2k}P_{2k}(0) \Bigr] + 
\frac{1}{2}a^2\om^2.
\end{equation}
In non-dimensional form the implicit equation for the equilibrium surface is
\begin{equation}\label{eq:Maclaurin_sph_surface}
\frac{1}{\xi}\Bigl[1 - \sum_{k=1}^{\inf} \xi^{-2k} J_{2k} P_{2k}(\mu)\Bigr] + 
\frac{q}{3}\xi^2\bigl[1 - P_2(\mu)\bigl] - \frac{q}{2} - 1 +
\sum_{k=1}^{\inf}J_{2k}P_{2k}(0) = 0.
\end{equation}

We have two sets of coupled equations. Equation~(\ref{eq:Maclaurin_sph_surface})
for the shape of the surface given the gravity coefficients $J_n$, and equations 
\eqref{eq:Jn_maclaurin_ext} for the gravity coefficients given the level surface
$\xi(\mu)$. We proceed with an iterative solution. Given a value of $q$ and
initial guesses for $J_n$ and/or $\xi$ we numerically integrate
eqs.~\eqref{eq:Jn_maclaurin_ext}. When the integration routine requires the value
of $\xi$ at some point $\mu'$ we obtain it by numerically solving
eq.~\eqref{eq:Maclaurin_sph_surface}, using the current values $J_n$. This results
in updated values of $J_n$. We iterate this procedure until all $J_n$ values
converge to the chosen tolerance, usually to machine precision.

A side note on practical implementation. The numerical integration of
(\ref{eq:Jn_maclaurin_ext}) can be done in any way we prefer but in practice it is
best to use a scheme that evaluates the integrand at a constant number of fixed
points. Gauss-Legendre integration works well because $\xi(\mu)$ is expected to be
a low-order polynomial, and a relatively manageable number of abscissas,
$\mu_\alpha$, $\alpha=1,2,\ldots,L$ are needed to yield precise results.

% section a_constant_density_spheroid (end)

% References
\bibliographystyle{plainnat} % [plainnat|abbrvnat|agu08|agufull08|...]
\bibliography{/users/naor/documents/library}

% ----------------------------------------------------------------
\end{document}
% ----------------------------------------------------------------
%END OF FILE
%-----------------------------------------------------------------
